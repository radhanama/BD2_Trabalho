\documentclass[12pt]{article}
\usepackage{pdflscape}
\usepackage{listings}
\usepackage{sbc-template} 
\usepackage{graphicx,url}
\usepackage{url}
\usepackage[brazil]{babel} 
\usepackage[utf8]{inputenc} 
\usepackage[T1]{fontenc}
\usepackage[normalem]{ulem}
\usepackage[hidelinks]{hyperref}
\usepackage{svg}

\usepackage[square,authoryear]{natbib}
\usepackage{amssymb} 
\usepackage{mathalfa} 
\usepackage{algorithm} 
\usepackage{algpseudocode} 
\usepackage[table]{xcolor}
\usepackage{array}
\usepackage{titlesec}
\usepackage{mdframed}
\usepackage{listings}

\usepackage{amsmath} 
\usepackage{booktabs}

\urlstyle{same}

\newcolumntype{L}[1]{>{\raggedright\let\newline\\\arraybackslash\hspace{0pt}}m{#1}}
\newcolumntype{C}[1]{>{\centering\let\newline\\\arraybackslash\hspace{0pt}}m{#1}}
\newcolumntype{R}[1]{>{\raggedleft\let\newline\\\arraybackslash\hspace{0pt}}m{#1}}

\newcommand\Tstrut{\rule{0pt}{2.6ex}} 
\newcommand\Bstrut{\rule[-0.9ex]{0pt}{0pt}} 
\newcommand{\scell}[2][c]{\begin{tabular}[#1]{@{}c@{}}#2\end{tabular}}

\usepackage[nolist,nohyperlinks]{acronym}

\title{Trabalho de Administração de Banco de Dados}

\author{Radhanama Mesiano\inst{1}}


\address{Centro Federal de Educação Tecnológica Celso Suckow da Fonseca - CEFET/RJ
	\email{radhanama.mesiano@aluno.cefet-rj.br}
}
\lstdefinestyle{sql}{
  language=SQL,
  basicstyle=\ttfamily\small,
  keywordstyle=\bfseries,
  commentstyle=\itshape\color{gray},
  numbers=left,
  numberstyle=\tiny\color{gray},
  frame=single,
  breaklines=true,
  breakatwhitespace=true,
  showstringspaces=false,
  captionpos=b,
}

\lstset{
  language=SQL,
  breaklines=true, % quebra de linha automática
}
\begin{document}

\begin{landscape}
\maketitle

\section*{Introdução}
Neste documento, conduzimos uma análise abrangente de dez consultas do TPC-H 2.6.1, abordando inicialmente as cinco consultas originais do benchmark, seguidas por cinco consultas adicionais desenvolvidas exclusivamente para este relatório. Este relatorio detalhado proporcionará uma compreensão aprofundada do desempenho e da eficácia das consultas, destacando eventuais otimizações e estratégias implementadas para alcançar resultados mais eficientes.
\section{Primeira Consulta:}


\subsection{Objetivo da consulta:}
Esta consulta determina o valor dos produtos enviados entre certas nações para auxiliar na renegociação de contratos de envio.

\subsection{Consulta em SQL:}
\begin{lstlisting}
SELECT SUPP_NATION,
	CUST_NATION,
	L_YEAR,
	SUM(VOLUME) AS REVENUE
FROM (
		SELECT N1.N_NAME AS SUPP_NATION,
			N2.N_NAME AS CUST_NATION,
			EXTRACT(
				YEAR
				FROM L_SHIPDATE
			) AS L_YEAR,
			L_EXTENDEDPRICE * (1 - L_DISCOUNT) AS VOLUME
		FROM SUPPLIER,
			LINEITEM,
			ORDERS,
			CUSTOMER,
			NATION N1,
			NATION N2
		WHERE S_SUPPKEY = L_SUPPKEY
			AND O_ORDERKEY = L_ORDERKEY
			AND C_CUSTKEY = O_CUSTKEY
			AND S_NATIONKEY = N1.N_NATIONKEY
			AND C_NATIONKEY = N2.N_NATIONKEY
			AND (
				(
					N1.N_NAME = 'FRANCE'
					AND N2.N_NAME = 'GERMANY'
				)
				OR (
					N1.N_NAME = 'GERMANY'
					AND N2.N_NAME = 'FRANCE'
				)
			)
			AND L_SHIPDATE BETWEEN date '1995-01-01' AND date '1996-12-31'
	) AS SHIPPING
GROUP BY SUPP_NATION,
	CUST_NATION,
	L_YEAR
ORDER BY SUPP_NATION,
	CUST_NATION,
	L_YEAR;
\end{lstlisting}
\subsection{Tabela com os tempos das consultas e a média.}
\begin{tabular}{|c|c|}
  \hline
  Execução 1 & 00:00:00.252 \\
  Execução 2 & 00:00:00.522 \\
  Execução 3 & 00:00:00.205 \\
  Execução 4 & 00:00:00.186 \\
  Execução 5 & 00:00:00.249 \\
  \hline
  Média & 00:00:00.282 \\
  \hline
\end{tabular}

\subsection{Plano de Consulta gerado pelo postgresql:}
\lstinputlisting[style=sql, caption=Plano de consulta, label=lst:explain]{./planos/1.csv}
\subsection{Ações que pode melhorar o desempenho das 
consultas:} Criação de Índice: Certifique-se de ter índices nas colunas usadas nas condições de junção, como \texttt{S\_SUPPKEY}, \texttt{L\_SUPPKEY}, \texttt{O\_ORDERKEY}, \texttt{C\_CUSTKEY}, \texttt{N\_NATIONKEY} e \texttt{L\_SHIPDATE}.

\section{Segunda Consulta:}


\subsection{Objetivo da consulta:}
Esta consulta determina como a participação de mercado de uma nação específica dentro de uma região específica mudou ao longo de dois anos para um determinado tipo de peça.

\subsection{Consulta em SQL:}
\begin{lstlisting}
SELECT O_YEAR,
	SUM(
		CASE
			WHEN NATION = 'BRAZIL' THEN VOLUME
			ELSE 0
		END
	) / SUM(VOLUME) AS MKT_SHARE
FROM (
		SELECT EXTRACT(
				YEAR
				FROM O_ORDERDATE
			) AS O_YEAR,
			L_EXTENDEDPRICE * (1 - L_DISCOUNT) AS VOLUME,
			N2.N_NAME AS NATION
		FROM PART,
			SUPPLIER,
			LINEITEM,
			ORDERS,
			CUSTOMER,
			NATION N1,
			NATION N2,
			REGION
		WHERE P_PARTKEY = L_PARTKEY
			AND S_SUPPKEY = L_SUPPKEY
			AND L_ORDERKEY = O_ORDERKEY
			AND O_CUSTKEY = C_CUSTKEY
			AND C_NATIONKEY = N1.N_NATIONKEY
			AND N1.N_REGIONKEY = R_REGIONKEY
			AND R_NAME = 'AMERICA'
			AND S_NATIONKEY = N2.N_NATIONKEY
			AND O_ORDERDATE BETWEEN date '1995-01-01' AND date '1996-12-31'
			AND P_TYPE = 'ECONOMY ANODIZED STEEL'
	) AS ALL_NATIONS
GROUP BY O_YEAR
ORDER BY O_YEAR;
\end{lstlisting}

\subsection{Tabela com os tempos das consultas e a média.}
\begin{tabular}{|c|c|}
  \hline
  Execução 1 & 00:00:00.133 \\
  Execução 2 & 00:00:00.439 \\
  Execução 3 & 00:00:00.155 \\
  Execução 4 & 00:00:00.124 \\
  Execução 5 & 00:00:00.120 \\
  \hline
  Média & 00:00:00.194 \\
  \hline
\end{tabular}

\subsection{Plano de Consulta gerado pelo postgresql:}
\lstinputlisting[style=sql, caption=Plano de consulta, label=lst:explain]{./planos/2.csv}
\subsection{Ações que pode melhorar o desempenho das 
consultas:} Criação de Índice: Certifique-se de ter índices nas colunas usadas nas condições de junção, como \texttt{P\_PARTKEY}, \texttt{L\_PARTKEY}, \texttt{S\_SUPPKEY}, \texttt{L\_SUPPKEY}, \texttt{O\_ORDERKEY}, \texttt{C\_CUSTKEY}, \texttt{N\_NATIONKEY} e \texttt{O\_ORDERDATE}.

\section{Terceira Consulta:}


\subsection{Objetivo da consulta:}
Esta consulta verifica quantos fornecedores podem fornecer peças com atributos específicos. Pode ser usada, por exemplo, para determinar se há um número suficiente de fornecedores para peças com alta demanda.

\subsection{Consulta em SQL:}
\begin{lstlisting}
select p_brand,
	p_type,
	p_size,
	count(distinct ps_suppkey) as supplier_cnt
from partsupp,
	part
where p_partkey = ps_partkey
	and p_brand <> 'Brand#45'
	and p_type not like 'MEDIUM POLISHED%'
	and p_size in (49, 14, 23, 45, 19, 3, 36, 9)
	and ps_suppkey not in (
		select s_suppkey
		from supplier
		where s_comment like '%Customer%Complaints%'
	)
group by p_brand,
	p_type,
	p_size
order by supplier_cnt desc,
	p_brand,
	p_type,
	p_size;
\end{lstlisting}

\subsection{Tabela com os tempos das consultas e a média.}
\begin{tabular}{|c|c|}
  \hline
  Execução 1 & 00:00:00.273 \\
  Execução 2 & 00:00:00.551 \\
  Execução 3 & 00:00:00.277 \\
  Execução 4 & 00:00:00.302 \\
  Execução 5 & 00:00:00.294 \\
  \hline
  Média & 00:00:00.339 \\
  \hline
\end{tabular}

\subsection{Plano de Consulta gerado pelo postgresql:}
\lstinputlisting[style=sql, caption=Plano de consulta, label=lst:explain]{./planos/3.csv}
\subsection{Ações que pode melhorar o desempenho das 
consultas:} Criação de Índice: Certifique-se de ter índices em \texttt{PARTSUPP.PS\_PARTKEY}, \texttt{PARTSUPP.PS\_SUPPKEY} e \texttt{PART.P\_PARTKEY} para otimizar as condições de junção.
\section{Quarta Consulta:}


\subsection{Objetivo da consulta:}
A Consulta do Cliente de Grande Volume classifica os clientes com base em terem feito um pedido de grande quantidade. Pedidos de grande quantidade são definidos como aqueles cuja quantidade total está acima de um certo nível.

\subsection{Consulta em SQL:}
\begin{lstlisting}
select c_name,
	c_custkey,
	o_orderkey,
	o_orderdate,
	o_totalprice,
	sum(l_quantity)
from customer,
	orders,
	lineitem
where o_orderkey in (
		select l_orderkey
		from lineitem
		group by l_orderkey
		having sum(l_quantity) > 300
	)
	and c_custkey = o_custkey
	and o_orderkey = l_orderkey
group by c_name,
	c_custkey,
	o_orderkey,
	o_orderdate,
	o_totalprice
order by o_totalprice desc,
	o_orderdate;
\end{lstlisting}

\subsection{Tabela com os tempos das consultas e a média.}
\begin{tabular}{|c|c|}
  \hline
  Execução 1 & 00:00:00.346 \\
  Execução 2 & 00:00:00.699 \\
  Execução 3 & 00:00:00.319 \\
  Execução 4 & 00:00:00.320 \\
  Execução 5 & 00:00:00.282 \\
  \hline
  Média & 00:00:00.393 \\
  \hline
\end{tabular}
\subsection{Plano de Consulta gerado pelo postgresql:}
\lstinputlisting[style=sql, caption=Plano de consulta, label=lst:explain]{./planos/4.csv}
\subsection{Ações que pode melhorar o desempenho das 
consultas:} Criação de Índice: Certifique-se de ter índices nas colunas usadas nas condições de junção, como \texttt{C\_CUSTKEY}, \texttt{O\_ORDERKEY} e \texttt{L\_ORDERKEY}.
\section{Quinta Consulta:}


\subsection{Objetivo da consulta:}
A Consulta de Receita com Desconto relata a receita bruta com desconto atribuída à venda de peças selecionadas manipuladas de uma maneira específica. Essa consulta é um exemplo de código que poderia ser gerado programaticamente por uma ferramenta de mineração de dados.

\subsection{Consulta em SQL:}
\begin{lstlisting}
select sum(l_extendedprice * (1 - l_discount)) as revenue
from lineitem,
	part
where (
		p_partkey = l_partkey
		and p_brand = 'Brand#12'
		and p_container in (
			'SM CASE', 'SM BOX', 'SM PACK', 'SM PKG')
			and l_quantity >= 1
			and l_quantity <= 1 + 10
			and p_size between 1 and 5
			and l_shipmode in ('AIR ', 'AIR REG')
			and l_shipinstruct = 'DELIVER IN PERSON'
		)
		or (
			p_partkey = l_partkey
			and p_brand = 'Brand#23'
			and p_container in ('MED BAG', 'MED BOX', 'MED PKG', 'MED PACK')
			and l_quantity >= 10
			and l_quantity <= 10 + 10
			and p_size between 1 and 10
			and l_shipmode in ('AIR', 'AIR REG')
			and l_shipinstruct = 'DELIVER IN PERSON'
		)
		or (
			p_partkey = l_partkey
			and p_brand = 'Brand#34'
			and p_container in (
				'LG CASE', 'LG BOX', 'LG PACK', 'LG PKG')
				and l_quantity >= 20
				and l_quantity <= 20 + 10
				and p_size between 1 and 15
				and l_shipmode in ('AIR', 'AIR REG')
				and l_shipinstruct = 'DELIVER IN PERSON'
			);
\end{lstlisting}

\subsection{Tabela com os tempos das consultas e a média.}
\begin{tabular}{|c|c|}
  \hline
  Execução 1 & 00:00:00.124 \\
  Execução 2 & 00:00:00.413 \\
  Execução 3 & 00:00:00.123 \\
  Execução 4 & 00:00:00.143 \\
  Execução 5 & 00:00:00.122 \\
  \hline
  Média & 00:00:00.185 \\
  \hline
\end{tabular}

\subsection{Plano de Consulta gerado pelo postgresql:}
\lstinputlisting[style=sql, caption=Plano de consulta, label=lst:explain]{./planos/5.csv}
\subsection{Ações que pode melhorar o desempenho das 
consultas:} Criação de Índice: Certifique-se de ter índices nas colunas usadas nas condições de junção, como \texttt{p\_partkey} e \texttt{l\_partkey}. Além disso, índices nas colunas de filtragem como \texttt{p\_brand}, \texttt{p\_container}, \texttt{l\_quantity}, \texttt{p\_size}, \texttt{l\_shipmode} e \texttt{l\_shipinstruct} podem melhorar o desempenho.

\section{Sexta Consulta}


\subsection{Objetivo da consulta:}
Encontrar os 10 principais clientes com a maior receita total

\subsection{Consulta em SQL:}
\begin{lstlisting}
WITH CustomerRevenue AS (
    SELECT
        C_CUSTKEY,
        C_NAME,
        SUM(L_EXTENDEDPRICE * (1 - L_DISCOUNT)) AS TotalRevenue
    FROM
        CUSTOMER
        JOIN ORDERS ON CUSTOMER.C_CUSTKEY = ORDERS.O_CUSTKEY
        JOIN LINEITEM ON ORDERS.O_ORDERKEY = LINEITEM.L_ORDERKEY
    GROUP BY
        C_CUSTKEY,
        C_NAME
)
SELECT
    C_NAME,
    TotalRevenue
FROM
    CustomerRevenue
ORDER BY
    TotalRevenue DESC
LIMIT 10;
\end{lstlisting}

\subsection{Tabela com os tempos das consultas e a média.}
\begin{tabular}{|c|c|}
  \hline
  Execução 1 & 00:00:00.357 \\
  Execução 2 & 00:00:00.614 \\
  Execução 3 & 00:00:00.302 \\
  Execução 4 & 00:00:00.333 \\
  Execução 5 & 00:00:00.420 \\
  \hline
  Média & 00:00:00.405 \\
  \hline
\end{tabular}

\subsection{Plano de Consulta gerado pelo postgresql:}
\lstinputlisting[style=sql, caption=Plano de consulta, label=lst:explain]{./planos/6.csv}
\subsection{Ações que pode melhorar o desempenho das 
consultas:} Criação de Índice: Certifique-se de ter índices nas colunas usadas nas condições de junção, como \texttt{C\_CUSTKEY} em CUSTOMER e \texttt{O\_CUSTKEY} em ORDERS.
\section{Sétima Consulta}


\subsection{Objetivo da consulta:}
Calcular a média móvel dos preços totais do pedido para cada cliente

\subsection{Consulta em SQL:}
\begin{lstlisting}
WITH CustomerOrderTotals AS (
    SELECT
        O_CUSTKEY,
        O_ORDERKEY,
        O_TOTALPRICE,
        AVG(O_TOTALPRICE) OVER (PARTITION BY O_CUSTKEY ORDER BY O_ORDERKEY) AS RollingAvgTotalPrice
    FROM
        ORDERS
)
SELECT
    C_NAME,
    O_ORDERKEY,
    O_TOTALPRICE,
    RollingAvgTotalPrice
FROM
    CustomerOrderTotals
    JOIN CUSTOMER ON CustomerOrderTotals.O_CUSTKEY = CUSTOMER.C_CUSTKEY;

\end{lstlisting}

\subsection{Tabela com os tempos das consultas e a média.}
\begin{tabular}{|c|c|}
  \hline
  Execução 1 & 00:00:00.945 \\
  Execução 2 & 00:00:01.558 \\
  Execução 3 & 00:00:01.211 \\
  Execução 4 & 00:00:01.057 \\
  Execução 5 & 00:00:01.277 \\
  \hline
  Média & 00:00:01.209 \\
  \hline
\end{tabular}

\subsection{Plano de Consulta gerado pelo postgresql:}
\lstinputlisting[style=sql, caption=Plano de consulta, label=lst:explain]{./planos/7.csv}
\subsection{Ações que pode melhorar o desempenho das 
consultas:} Criação de Índice: Certifique-se de ter índices nas colunas usadas nas condições de junção, como \texttt{O\_CUSTKEY}, \texttt{O\_ORDERKEY} e \texttt{L\_ORDERKEY}.
\section{Oitava Consulta}


\subsection{Objetivo da consulta:}
Calcular a receita total para cada região e ano

\subsection{Consulta em SQL:}
\begin{lstlisting}
SELECT
    R_NAME AS REGION,
    EXTRACT(YEAR FROM O_ORDERDATE) AS ORDER_YEAR,
    SUM(L_EXTENDEDPRICE * (1 - L_DISCOUNT)) AS TOTAL_REVENUE
FROM
    REGION,
    NATION,
    CUSTOMER,
    ORDERS,
    LINEITEM
WHERE
    REGION.R_REGIONKEY = NATION.N_REGIONKEY
    AND NATION.N_NATIONKEY = CUSTOMER.C_NATIONKEY
    AND CUSTOMER.C_CUSTKEY = ORDERS.O_CUSTKEY
    AND ORDERS.O_ORDERKEY = LINEITEM.L_ORDERKEY
GROUP BY
    R_NAME,
    EXTRACT(YEAR FROM O_ORDERDATE)
ORDER BY
    REGION,
    ORDER_YEAR;
\end{lstlisting}

\subsection{Tabela com os tempos das consultas e a média.}
\begin{tabular}{|c|c|}
  \hline
  Execução 1 & 00:00:00.555 \\
  Execução 2 & 00:00:00.776 \\
  Execução 3 & 00:00:00.456 \\
  Execução 4 & 00:00:00.461 \\
  Execução 5 & 00:00:00.461 \\
  \hline
  Média & 00:00:00.542 \\
  \hline
\end{tabular}

\subsection{Plano de Consulta gerado pelo postgresql:}
\lstinputlisting[style=sql, caption=Plano de consulta, label=lst:explain]{./planos/8.csv}
\subsection{Ações que pode melhorar o desempenho das 
consultas:} Criação de Índice: Certifique-se de ter índices nas colunas usadas nas condições de junção, como \texttt{REGION.R\_REGIONKEY}, \texttt{NATION.N\_REGIONKEY}, \texttt{CUSTOMER.C\_NATIONKEY}, \texttt{ORDERS.O\_CUSTKEY} e \texttt{LINEITEM.L\_ORDERKEY}.
\section{Nona Consulta}


\subsection{Objetivo da consulta:}
Encontrar os clientes que fizeram pedidos de pelo menos três tipos de produtos diferentes

\subsection{Consulta em SQL:}
\begin{lstlisting}
SELECT
    C_NAME,
    COUNT(DISTINCT P_TYPE) AS DISTINCT_PRODUCT_TYPES
FROM
    CUSTOMER,
    ORDERS,
    LINEITEM,
    PART
WHERE
    CUSTOMER.C_CUSTKEY = ORDERS.O_CUSTKEY
    AND ORDERS.O_ORDERKEY = LINEITEM.L_ORDERKEY
    AND LINEITEM.L_PARTKEY = PART.P_PARTKEY
GROUP BY
    C_NAME
HAVING
    COUNT(DISTINCT P_TYPE) >= 3
ORDER BY
    DISTINCT_PRODUCT_TYPES DESC,
    C_NAME;

\end{lstlisting}

\subsection{Tabela com os tempos das consultas e a média.}
\begin{tabular}{|c|c|}
  \hline
  Execução 1 & 00:00:02.538 \\
  Execução 2 & 00:00:02.839 \\
  Execução 3 & 00:00:02.443 \\
  Execução 4 & 00:00:02.469 \\
  Execução 5 & 00:00:02.433 \\
  \hline
  Média & 00:00:02.544 \\
  \hline
\end{tabular}

\subsection{Plano de Consulta gerado pelo postgresql:}
\lstinputlisting[style=sql, caption=Plano de consulta, label=lst:explain]{./planos/9.csv}
\subsection{Ações que pode melhorar o desempenho das 
consultas:} Criação de Índice: Certifique-se de ter índices nas colunas usadas nas condições de junção, como \texttt{CUSTOMER.C\_CUSTKEY}, \texttt{ORDERS.O\_ORDERKEY} e \texttt{LINEITEM.L\_PARTKEY}.

\section{Décima Consulta}


\subsection{Objetivo da consulta:}
Identificar fornecedores que não forneceram peças para uma marca específica

\subsection{Consulta em SQL:}
\begin{lstlisting}
SELECT
    S_SUPPKEY,
    S_NAME,
    S_NATIONKEY
FROM
    SUPPLIER
WHERE
    S_SUPPKEY NOT IN (
        SELECT DISTINCT PS_SUPPKEY
        FROM PARTSUPP,
             PART
        WHERE PARTSUPP.PS_PARTKEY = PART.P_PARTKEY
          AND P_BRAND = 'Brand#12'
    )
ORDER BY
    S_SUPPKEY;
\end{lstlisting}

\subsection{Tabela com os tempos das consultas e a média.}
\begin{tabular}{|c|c|}
  \hline
  Execução 1 & 00:00:00.221 \\
  Execução 2 & 00:00:00.498 \\
  Execução 3 & 00:00:00.204 \\
  Execução 4 & 00:00:00.237 \\
  Execução 5 & 00:00:00.288 \\
  \hline
  Média & 00:00:00.290 \\
  \hline
\end{tabular}

\subsection{Plano de Consulta gerado pelo postgresql:}
\lstinputlisting[style=sql, caption=Plano de consulta, label=lst:explain]{./planos/10.csv}
\subsection{Ações que pode melhorar o desempenho das 
consultas:} Criação de Índice: Certifique-se de ter índices nas colunas usadas nas condições de junção, como \texttt{SUPPLIER.S\_SUPPKEY}, \texttt{PARTSUPP.PS\_SUPPKEY} e \texttt{PARTSUPP.PS\_PARTKEY}.

\end{landscape}

\end{document}
